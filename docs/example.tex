\documentclass{article}

\usepackage{comment, multicol}
\usepackage{hyperref}
\usepackage{calc,pict2e,picture}
\usepackage{textgreek,textcomp,gensymb,stix}

\setcounter{secnumdepth}{2}

\title{\LaTeX $\varheartsuit$ Post}
\author{made with $\varheartsuit$ by RogueAI}
\date{2021}


\begin{document}
    \maketitle
    \begin{abstract}
        This document will show most of the features of \LaTeX.js while at the same time being a gentle introduction to \LaTeX.
        In the appendix, the API as well as the format of custom macro definitions in JavaScript will be explained.
    \end{abstract}
    
    \section{Characters}

    It is possible to input any UTF-8 character either directly or by character code
    using one of the following:

    \begin{itemize}
        \item \texttt{\textbackslash symbol\{"00A9\}}: \symbol{"00A9}
        \item \verb|\char"A9|: \char"A9
        \item \verb|^^A9 or ^^^^00A9|: ^^A9 or ^^^^00A9
    \end{itemize}

    \bigskip

    \noindent
    Special characters, like those:
    \begin{center}
        \$ \& \% \# \_ \{ \} \~{} \^{} \textbackslash % \< \>  \"   % TODO cannot be typeset
    \end{center}
%
    have to be escaped.

    More than 200 symbols are accessible through macros. For instance: 30\,\textcelsius{} is
    86\,\textdegree{}F. See also Section~\ref{sec:symbols}.

\end{document}
